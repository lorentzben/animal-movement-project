% Options for packages loaded elsewhere
\PassOptionsToPackage{unicode}{hyperref}
\PassOptionsToPackage{hyphens}{url}
%
\documentclass[
]{article}
\usepackage{amsmath,amssymb}
\usepackage{iftex}
\ifPDFTeX
  \usepackage[T1]{fontenc}
  \usepackage[utf8]{inputenc}
  \usepackage{textcomp} % provide euro and other symbols
\else % if luatex or xetex
  \usepackage{unicode-math} % this also loads fontspec
  \defaultfontfeatures{Scale=MatchLowercase}
  \defaultfontfeatures[\rmfamily]{Ligatures=TeX,Scale=1}
\fi
\usepackage{lmodern}
\ifPDFTeX\else
  % xetex/luatex font selection
\fi
% Use upquote if available, for straight quotes in verbatim environments
\IfFileExists{upquote.sty}{\usepackage{upquote}}{}
\IfFileExists{microtype.sty}{% use microtype if available
  \usepackage[]{microtype}
  \UseMicrotypeSet[protrusion]{basicmath} % disable protrusion for tt fonts
}{}
\makeatletter
\@ifundefined{KOMAClassName}{% if non-KOMA class
  \IfFileExists{parskip.sty}{%
    \usepackage{parskip}
  }{% else
    \setlength{\parindent}{0pt}
    \setlength{\parskip}{6pt plus 2pt minus 1pt}}
}{% if KOMA class
  \KOMAoptions{parskip=half}}
\makeatother
\usepackage{xcolor}
\usepackage[margin=1in]{geometry}
\usepackage{color}
\usepackage{fancyvrb}
\newcommand{\VerbBar}{|}
\newcommand{\VERB}{\Verb[commandchars=\\\{\}]}
\DefineVerbatimEnvironment{Highlighting}{Verbatim}{commandchars=\\\{\}}
% Add ',fontsize=\small' for more characters per line
\usepackage{framed}
\definecolor{shadecolor}{RGB}{248,248,248}
\newenvironment{Shaded}{\begin{snugshade}}{\end{snugshade}}
\newcommand{\AlertTok}[1]{\textcolor[rgb]{0.94,0.16,0.16}{#1}}
\newcommand{\AnnotationTok}[1]{\textcolor[rgb]{0.56,0.35,0.01}{\textbf{\textit{#1}}}}
\newcommand{\AttributeTok}[1]{\textcolor[rgb]{0.13,0.29,0.53}{#1}}
\newcommand{\BaseNTok}[1]{\textcolor[rgb]{0.00,0.00,0.81}{#1}}
\newcommand{\BuiltInTok}[1]{#1}
\newcommand{\CharTok}[1]{\textcolor[rgb]{0.31,0.60,0.02}{#1}}
\newcommand{\CommentTok}[1]{\textcolor[rgb]{0.56,0.35,0.01}{\textit{#1}}}
\newcommand{\CommentVarTok}[1]{\textcolor[rgb]{0.56,0.35,0.01}{\textbf{\textit{#1}}}}
\newcommand{\ConstantTok}[1]{\textcolor[rgb]{0.56,0.35,0.01}{#1}}
\newcommand{\ControlFlowTok}[1]{\textcolor[rgb]{0.13,0.29,0.53}{\textbf{#1}}}
\newcommand{\DataTypeTok}[1]{\textcolor[rgb]{0.13,0.29,0.53}{#1}}
\newcommand{\DecValTok}[1]{\textcolor[rgb]{0.00,0.00,0.81}{#1}}
\newcommand{\DocumentationTok}[1]{\textcolor[rgb]{0.56,0.35,0.01}{\textbf{\textit{#1}}}}
\newcommand{\ErrorTok}[1]{\textcolor[rgb]{0.64,0.00,0.00}{\textbf{#1}}}
\newcommand{\ExtensionTok}[1]{#1}
\newcommand{\FloatTok}[1]{\textcolor[rgb]{0.00,0.00,0.81}{#1}}
\newcommand{\FunctionTok}[1]{\textcolor[rgb]{0.13,0.29,0.53}{\textbf{#1}}}
\newcommand{\ImportTok}[1]{#1}
\newcommand{\InformationTok}[1]{\textcolor[rgb]{0.56,0.35,0.01}{\textbf{\textit{#1}}}}
\newcommand{\KeywordTok}[1]{\textcolor[rgb]{0.13,0.29,0.53}{\textbf{#1}}}
\newcommand{\NormalTok}[1]{#1}
\newcommand{\OperatorTok}[1]{\textcolor[rgb]{0.81,0.36,0.00}{\textbf{#1}}}
\newcommand{\OtherTok}[1]{\textcolor[rgb]{0.56,0.35,0.01}{#1}}
\newcommand{\PreprocessorTok}[1]{\textcolor[rgb]{0.56,0.35,0.01}{\textit{#1}}}
\newcommand{\RegionMarkerTok}[1]{#1}
\newcommand{\SpecialCharTok}[1]{\textcolor[rgb]{0.81,0.36,0.00}{\textbf{#1}}}
\newcommand{\SpecialStringTok}[1]{\textcolor[rgb]{0.31,0.60,0.02}{#1}}
\newcommand{\StringTok}[1]{\textcolor[rgb]{0.31,0.60,0.02}{#1}}
\newcommand{\VariableTok}[1]{\textcolor[rgb]{0.00,0.00,0.00}{#1}}
\newcommand{\VerbatimStringTok}[1]{\textcolor[rgb]{0.31,0.60,0.02}{#1}}
\newcommand{\WarningTok}[1]{\textcolor[rgb]{0.56,0.35,0.01}{\textbf{\textit{#1}}}}
\usepackage{graphicx}
\makeatletter
\def\maxwidth{\ifdim\Gin@nat@width>\linewidth\linewidth\else\Gin@nat@width\fi}
\def\maxheight{\ifdim\Gin@nat@height>\textheight\textheight\else\Gin@nat@height\fi}
\makeatother
% Scale images if necessary, so that they will not overflow the page
% margins by default, and it is still possible to overwrite the defaults
% using explicit options in \includegraphics[width, height, ...]{}
\setkeys{Gin}{width=\maxwidth,height=\maxheight,keepaspectratio}
% Set default figure placement to htbp
\makeatletter
\def\fps@figure{htbp}
\makeatother
\setlength{\emergencystretch}{3em} % prevent overfull lines
\providecommand{\tightlist}{%
  \setlength{\itemsep}{0pt}\setlength{\parskip}{0pt}}
\setcounter{secnumdepth}{-\maxdimen} % remove section numbering
\ifLuaTeX
  \usepackage{selnolig}  % disable illegal ligatures
\fi
\IfFileExists{bookmark.sty}{\usepackage{bookmark}}{\usepackage{hyperref}}
\IfFileExists{xurl.sty}{\usepackage{xurl}}{} % add URL line breaks if available
\urlstyle{same}
\hypersetup{
  pdftitle={Initial Analysis},
  pdfauthor={Ben Lorentz},
  hidelinks,
  pdfcreator={LaTeX via pandoc}}

\title{Initial Analysis}
\author{Ben Lorentz}
\date{2023-09-30}

\begin{document}
\maketitle

\begin{enumerate}
\def\labelenumi{\arabic{enumi}.}
\tightlist
\item
  Descriptive statistics
\end{enumerate}

\begin{enumerate}
\def\labelenumi{\alph{enumi}.}
\tightlist
\item
  Read in Data
\end{enumerate}

\begin{Shaded}
\begin{Highlighting}[]
\FunctionTok{source}\NormalTok{(}\StringTok{".installpackages.R"}\NormalTok{)}
\end{Highlighting}
\end{Shaded}

\begin{verbatim}
## Package:  plyr was already installed.
## Package:  dplyr was already installed.
## Package:  ggplot2 was already installed.
## Package:  gganimate was already installed.
## Package:  tidyverse was already installed.
## Package:  lubridate was already installed.
## Package:  knitr was already installed.
## Package:  png was already installed.
## Package:  grid was already installed.
## Package:  amt was already installed.
\end{verbatim}

\begin{verbatim}
## The legacy packages maptools, rgdal, and rgeos, underpinning the sp package,
## which was just loaded, will retire in October 2023.
## Please refer to R-spatial evolution reports for details, especially
## https://r-spatial.org/r/2023/05/15/evolution4.html.
## It may be desirable to make the sf package available;
## package maintainers should consider adding sf to Suggests:.
## The sp package is now running under evolution status 2
##      (status 2 uses the sf package in place of rgdal)
\end{verbatim}

\begin{verbatim}
## Package:  ctmm was already installed.
\end{verbatim}

\begin{verbatim}
## Registered S3 methods overwritten by 'adehabitatMA':
##   method                       from
##   print.SpatialPixelsDataFrame sp  
##   print.SpatialPixels          sp
\end{verbatim}

\begin{verbatim}
## Package:  adehabitatLT was already installed.
## Package:  circular was already installed.
## Package:  move2 was already installed.
## Package:  KernSmooth was already installed.
## Package:  sf was already installed.
## Package:  raster was already installed.
## Package:  terra was already installed.
## Package:  sp was already installed.
\end{verbatim}

\begin{verbatim}
## Please note that rgdal will be retired during October 2023,
## plan transition to sf/stars/terra functions using GDAL and PROJ
## at your earliest convenience.
## See https://r-spatial.org/r/2023/05/15/evolution4.html and https://github.com/r-spatial/evolution
## rgdal: version: 1.6-7, (SVN revision 1203)
## Geospatial Data Abstraction Library extensions to R successfully loaded
## Loaded GDAL runtime: GDAL 3.4.2, released 2022/03/08
## Path to GDAL shared files: /Users/benlorentz/Library/R/x86_64/4.2/library/rgdal/gdal
## GDAL binary built with GEOS: FALSE 
## Loaded PROJ runtime: Rel. 8.2.1, January 1st, 2022, [PJ_VERSION: 821]
## Path to PROJ shared files: /Users/benlorentz/Library/R/x86_64/4.2/library/rgdal/proj
## PROJ CDN enabled: FALSE
## Linking to sp version:1.6-1
## To mute warnings of possible GDAL/OSR exportToProj4() degradation,
## use options("rgdal_show_exportToProj4_warnings"="none") before loading sp or rgdal.
\end{verbatim}

\begin{verbatim}
## Package:  rgdal was already installed.
## Package:  pracma was already installed.
## Package:  adehabitatHR was already installed.
## Package:  igraph was already installed.
## Package:  ergm was already installed.
## Package:  network was already installed.
## Package:  vroom was already installed.
\end{verbatim}

\begin{Shaded}
\begin{Highlighting}[]
\FunctionTok{library}\NormalTok{(tidyverse)}
\end{Highlighting}
\end{Shaded}

\begin{verbatim}
## -- Attaching core tidyverse packages ------------------------ tidyverse 2.0.0 --
## v dplyr     1.1.3     v readr     2.1.4
## v forcats   1.0.0     v stringr   1.5.0
## v ggplot2   3.4.3     v tibble    3.2.1
## v lubridate 1.9.2     v tidyr     1.3.0
## v purrr     1.0.2
\end{verbatim}

\begin{verbatim}
## -- Conflicts ------------------------------------------ tidyverse_conflicts() --
## x dplyr::filter() masks stats::filter()
## x dplyr::lag()    masks stats::lag()
## i Use the conflicted package (<http://conflicted.r-lib.org/>) to force all conflicts to become errors
\end{verbatim}

\begin{Shaded}
\begin{Highlighting}[]
\FunctionTok{library}\NormalTok{(amt)}
\end{Highlighting}
\end{Shaded}

\begin{verbatim}
## 
## Attaching package: 'amt'
## 
## The following object is masked from 'package:stats':
## 
##     filter
\end{verbatim}

\begin{Shaded}
\begin{Highlighting}[]
\FunctionTok{library}\NormalTok{(sf)}
\end{Highlighting}
\end{Shaded}

\begin{verbatim}
## Linking to GEOS 3.10.2, GDAL 3.4.2, PROJ 8.2.1; sf_use_s2() is TRUE
\end{verbatim}

\begin{Shaded}
\begin{Highlighting}[]
\FunctionTok{library}\NormalTok{(lubridate) }
\FunctionTok{library}\NormalTok{(ggplot2)}
\FunctionTok{library}\NormalTok{(terra)}
\end{Highlighting}
\end{Shaded}

\begin{verbatim}
## terra 1.7.46
## 
## Attaching package: 'terra'
## 
## The following object is masked from 'package:tidyr':
## 
##     extract
\end{verbatim}

\begin{Shaded}
\begin{Highlighting}[]
\FunctionTok{library}\NormalTok{(png)}


\NormalTok{vulture\_metadata }\OtherTok{\textless{}{-}} \FunctionTok{read.csv}\NormalTok{(}\StringTok{"../data/Black{-}Vultures{-}and{-}Turkey{-}Vultures{-}Southeastern{-}USA{-}reference{-}data.csv"}\NormalTok{)}
\FunctionTok{print}\NormalTok{(vulture\_metadata)}
\end{Highlighting}
\end{Shaded}

\begin{verbatim}
##    tag.id animal.id     animal.taxon          deploy.on.date
## 1     171         6   Cathartes aura 2013-06-17 13:19:20.000
## 2     167         8 Coragyps atratus 2013-06-18 23:23:34.000
## 3     162        12 Coragyps atratus 2013-06-20 16:40:40.000
## 4     165        22 Coragyps atratus 2013-06-20 17:36:23.000
## 5     163         1   Cathartes aura 2013-06-26 14:07:33.000
## 6     173         3   Cathartes aura 2013-06-26 14:38:58.000
## 7     180        13   Cathartes aura 2013-06-27 18:39:55.000
## 8     161        47 Coragyps atratus 2013-07-03 16:06:48.000
## 9     177        48 Coragyps atratus 2013-07-03 18:51:03.000
## 10    172        60   Cathartes aura 2013-07-05 12:52:58.000
## 11    178        57 Coragyps atratus 2013-07-05 16:39:00.000
## 12    176        75   Cathartes aura 2013-07-09 15:31:45.000
## 13    166        90   Cathartes aura 2013-07-30 14:34:17.000
## 14    179        91   Cathartes aura 2013-07-31 15:52:51.000
## 15    175        92 Coragyps atratus 2013-08-02 12:47:47.000
## 16    168       108 Coragyps atratus 2014-04-21 18:40:25.000
## 17    178       123   Cathartes aura 2014-04-23 19:44:16.000
## 18    174       126 Coragyps atratus 2014-05-01 13:38:20.000
##            deploy.off.date animal.life.stage animal.sex attachment.type
## 1  2015-09-01 04:37:25.000             adult          m         harness
## 2  2014-06-18 22:02:41.000             adult          m         harness
## 3  2015-09-01 09:52:41.000             adult          m         harness
## 4  2015-09-01 03:07:07.000             adult          f         harness
## 5  2015-09-01 04:28:52.000             adult          f         harness
## 6  2015-09-01 04:52:35.000             adult          f         harness
## 7  2013-12-21 15:16:23.000             adult          f         harness
## 8  2015-09-01 04:31:54.000             adult          f         harness
## 9  2014-09-22 16:35:20.000             adult          m         harness
## 10 2015-09-01 04:58:46.000             adult          m         harness
## 11 2014-03-21 19:44:56.000             adult          f         harness
## 12 2015-09-01 04:46:12.000             adult          m         harness
## 13 2015-09-01 02:56:53.000             adult          m         harness
## 14 2015-09-01 04:55:41.000             adult          m         harness
## 15 2015-09-01 04:21:36.000             adult          f         harness
## 16 2015-09-01 04:20:15.000             adult          m         harness
## 17 2015-09-01 04:48:56.000             adult          m         harness
## 18 2015-09-01 04:54:57.000             adult          m         harness
##                                           deployment.end.comments
## 1                                          Active at end of study
## 2                         Transmission ceased; bird fate unknown.
## 3                                          Active at end of study
## 4                                                Vulture deceased
## 5                                          Active at end of study
## 6                                          Active at end of study
## 7                          Transmission ceased; bird fate unknown
## 8                                          Active at end of study
## 9  Transmitter dropped; bird fate unknown; transmitter redeployed
## 10                                         Active at end of study
## 11                              Deceased; transmitter redeployed.
## 12                                         Active at end of study
## 13                                         Active at end of study
## 14                                         Active at end of study
## 15                                         Active at end of study
## 16                                         Active at end of study
## 17                                         Active at end of study
## 18                                         Active at end of study
##    deployment.end.type deployment.id manipulation.type tag.manufacturer.name
## 1                other         171-6              none             Microwave
## 2              unknown         167-8              none             Microwave
## 3                other        162-12              none             Microwave
## 4                 dead        165-22              none             Microwave
## 5                other         163-1              none             Microwave
## 6                other         173-3              none             Microwave
## 7              unknown        180-13              none             Microwave
## 8                other        161-47              none             Microwave
## 9             fall-off        177-48              none             Microwave
## 10               other        172-60              none             Microwave
## 11                dead        178-57              none             Microwave
## 12               other        176-75              none             Microwave
## 13               other        166-90              none             Microwave
## 14               other        179-91              none             Microwave
## 15               other        175-92              none             Microwave
## 16               other       168-108              none             Microwave
## 17               other       178-123              none             Microwave
## 18               other       174-126              none             Microwave
##    tag.mass tag.readout.method
## 1        70      phone-network
## 2        70      phone-network
## 3        70      phone-network
## 4        70      phone-network
## 5        70      phone-network
## 6        70      phone-network
## 7        70      phone-network
## 8        70      phone-network
## 9        70      phone-network
## 10       70      phone-network
## 11       70      phone-network
## 12       70      phone-network
## 13       70      phone-network
## 14       70      phone-network
## 15       70      phone-network
## 16       70      phone-network
## 17       70      phone-network
## 18       70      phone-network
\end{verbatim}

\begin{Shaded}
\begin{Highlighting}[]
\NormalTok{vulture\_dat }\OtherTok{\textless{}{-}} \FunctionTok{read.csv}\NormalTok{(}\StringTok{"../data/Black{-}Vultures{-}and{-}Turkey{-}Vultures{-}Southeastern{-}USA.csv"}\NormalTok{)}
\FunctionTok{head}\NormalTok{(vulture\_dat)}
\end{Highlighting}
\end{Shaded}

\begin{verbatim}
##     event.id visible               timestamp location.long location.lat
## 1 3378871917    true 2013-06-18 23:26:49.000     -81.64039     33.15934
## 2 3378871918    true 2013-06-18 23:50:02.000     -81.64057     33.15920
## 3 3378871919    true 2013-06-19 00:14:17.000     -81.63882     33.15654
## 4 3378871920    true 2013-06-19 00:38:31.000     -81.63679     33.15425
## 5 3378871921    true 2013-06-19 01:02:46.000     -81.63669     33.15413
## 6 3378871922    true 2013-06-19 01:29:01.000     -81.63675     33.15424
##   gps.hdop gps.satellite.count gps.vdop ground.speed heading
## 1      3.0                   5      2.2            0       0
## 2      0.9                   9      1.4            0       0
## 3      0.9                   9      1.3            0     317
## 4      1.0                   8      1.3            0       0
## 5      1.4                   6      1.8            0       0
## 6      1.2                   6      1.6            0       0
##   height.above.ellipsoid manually.marked.outlier sensor.type
## 1                     80                      NA         gps
## 2                     79                      NA         gps
## 3                     73                      NA         gps
## 4                     72                      NA         gps
## 5                     76                      NA         gps
## 6                     48                      NA         gps
##   individual.taxon.canonical.name tag.local.identifier
## 1                Coragyps atratus                  167
## 2                Coragyps atratus                  167
## 3                Coragyps atratus                  167
## 4                Coragyps atratus                  167
## 5                Coragyps atratus                  167
## 6                Coragyps atratus                  167
##   individual.local.identifier
## 1                           8
## 2                           8
## 3                           8
## 4                           8
## 5                           8
## 6                           8
##                                            study.name
## 1 Black Vultures and Turkey Vultures Southeastern USA
## 2 Black Vultures and Turkey Vultures Southeastern USA
## 3 Black Vultures and Turkey Vultures Southeastern USA
## 4 Black Vultures and Turkey Vultures Southeastern USA
## 5 Black Vultures and Turkey Vultures Southeastern USA
## 6 Black Vultures and Turkey Vultures Southeastern USA
\end{verbatim}

\begin{Shaded}
\begin{Highlighting}[]
\FunctionTok{str}\NormalTok{(vulture\_dat)}
\end{Highlighting}
\end{Shaded}

\begin{verbatim}
## 'data.frame':    2605997 obs. of  17 variables:
##  $ event.id                       : num  3.38e+09 3.38e+09 3.38e+09 3.38e+09 3.38e+09 ...
##  $ visible                        : chr  "true" "true" "true" "true" ...
##  $ timestamp                      : chr  "2013-06-18 23:26:49.000" "2013-06-18 23:50:02.000" "2013-06-19 00:14:17.000" "2013-06-19 00:38:31.000" ...
##  $ location.long                  : num  -81.6 -81.6 -81.6 -81.6 -81.6 ...
##  $ location.lat                   : num  33.2 33.2 33.2 33.2 33.2 ...
##  $ gps.hdop                       : num  3 0.9 0.9 1 1.4 1.2 1.2 1.5 1.5 2.2 ...
##  $ gps.satellite.count            : int  5 9 9 8 6 6 7 7 7 5 ...
##  $ gps.vdop                       : num  2.2 1.4 1.3 1.3 1.8 1.6 1.7 2.3 1.7 2.2 ...
##  $ ground.speed                   : num  0 0 0 0 0 0 0 0 0 0 ...
##  $ heading                        : num  0 0 317 0 0 0 0 158 0 0 ...
##  $ height.above.ellipsoid         : num  80 79 73 72 76 48 83 59 66 72 ...
##  $ manually.marked.outlier        : logi  NA NA NA NA NA NA ...
##  $ sensor.type                    : chr  "gps" "gps" "gps" "gps" ...
##  $ individual.taxon.canonical.name: chr  "Coragyps atratus" "Coragyps atratus" "Coragyps atratus" "Coragyps atratus" ...
##  $ tag.local.identifier           : int  167 167 167 167 167 167 167 167 167 167 ...
##  $ individual.local.identifier    : int  8 8 8 8 8 8 8 8 8 8 ...
##  $ study.name                     : chr  "Black Vultures and Turkey Vultures Southeastern USA" "Black Vultures and Turkey Vultures Southeastern USA" "Black Vultures and Turkey Vultures Southeastern USA" "Black Vultures and Turkey Vultures Southeastern USA" ...
\end{verbatim}

\begin{Shaded}
\begin{Highlighting}[]
\FunctionTok{nrow}\NormalTok{(vulture\_dat)}
\end{Highlighting}
\end{Shaded}

\begin{verbatim}
## [1] 2605997
\end{verbatim}

\begin{Shaded}
\begin{Highlighting}[]
\CommentTok{\#vulture\_dat}
\end{Highlighting}
\end{Shaded}

\begin{enumerate}
\def\labelenumi{\alph{enumi}.}
\setcounter{enumi}{1}
\tightlist
\item
  how many individuals examined
\end{enumerate}

\begin{Shaded}
\begin{Highlighting}[]
\CommentTok{\# tag id\textquotesingle{}s 17 tags }

\FunctionTok{sort}\NormalTok{(}\FunctionTok{unique}\NormalTok{(vulture\_dat}\SpecialCharTok{$}\NormalTok{tag.local.identifier))}
\end{Highlighting}
\end{Shaded}

\begin{verbatim}
##  [1] 161 162 163 165 166 167 168 171 172 173 174 175 176 177 178 179 180
\end{verbatim}

\begin{Shaded}
\begin{Highlighting}[]
\FunctionTok{length}\NormalTok{(}\FunctionTok{unique}\NormalTok{(vulture\_dat}\SpecialCharTok{$}\NormalTok{tag.local.identifier))}
\end{Highlighting}
\end{Shaded}

\begin{verbatim}
## [1] 17
\end{verbatim}

\begin{Shaded}
\begin{Highlighting}[]
\CommentTok{\# individual bird id\textquotesingle{}s 18 birds}

\FunctionTok{sort}\NormalTok{(}\FunctionTok{unique}\NormalTok{(vulture\_dat}\SpecialCharTok{$}\NormalTok{individual.local.identifier))}
\end{Highlighting}
\end{Shaded}

\begin{verbatim}
##  [1]   1   3   6   8  12  13  22  47  48  57  60  75  90  91  92 108 123 126
\end{verbatim}

\begin{Shaded}
\begin{Highlighting}[]
\FunctionTok{length}\NormalTok{(}\FunctionTok{unique}\NormalTok{(vulture\_dat}\SpecialCharTok{$}\NormalTok{individual.local.identifier))}
\end{Highlighting}
\end{Shaded}

\begin{verbatim}
## [1] 18
\end{verbatim}

\begin{enumerate}
\def\labelenumi{\alph{enumi}.}
\setcounter{enumi}{2}
\tightlist
\item
  How many datapoints are present for each individual in the timeperiod
\end{enumerate}

\begin{Shaded}
\begin{Highlighting}[]
\NormalTok{bird\_ids }\OtherTok{\textless{}{-}} \FunctionTok{unique}\NormalTok{(vulture\_dat}\SpecialCharTok{$}\NormalTok{individual.local.identifier)}

\NormalTok{record\_table }\OtherTok{\textless{}{-}} \FunctionTok{data.frame}\NormalTok{()}

\ControlFlowTok{for}\NormalTok{(i }\ControlFlowTok{in} \DecValTok{1}\SpecialCharTok{:}\FunctionTok{length}\NormalTok{(bird\_ids))\{}
\NormalTok{  current\_bird }\OtherTok{\textless{}{-}}\NormalTok{ bird\_ids[i]}
\NormalTok{  number\_of\_records }\OtherTok{\textless{}{-}} \FunctionTok{sum}\NormalTok{(vulture\_dat}\SpecialCharTok{$}\NormalTok{individual.local.identifier }\SpecialCharTok{==}\NormalTok{ current\_bird)}
\NormalTok{  new\_row }\OtherTok{\textless{}{-}} \FunctionTok{c}\NormalTok{(current\_bird,number\_of\_records)}
  \CommentTok{\#colnames(new\_row) \textless{}{-} c("bird id","number of records")}
\NormalTok{  record\_table }\OtherTok{\textless{}{-}} \FunctionTok{rbind}\NormalTok{(record\_table, new\_row)}
\NormalTok{\}}

\FunctionTok{colnames}\NormalTok{(record\_table) }\OtherTok{\textless{}{-}}  \FunctionTok{c}\NormalTok{(}\StringTok{"bird id"}\NormalTok{,}\StringTok{"number of records"}\NormalTok{)}

\NormalTok{(record\_table)}
\end{Highlighting}
\end{Shaded}

\begin{verbatim}
##    bird id number of records
## 1        8             97255
## 2       12            168457
## 3       22            112667
## 4       47            113321
## 5       48             89945
## 6       57             29717
## 7      123            135554
## 8       92            172766
## 9      108             87646
## 10     126            123437
## 11       1            237897
## 12       3            243371
## 13       6            198840
## 14      13             38560
## 15      60            210787
## 16      75            188019
## 17      90            170442
## 18      91            187316
\end{verbatim}

\begin{enumerate}
\def\labelenumi{\alph{enumi}.}
\setcounter{enumi}{3}
\tightlist
\item
  When does each record start and stop?
\end{enumerate}

\begin{Shaded}
\begin{Highlighting}[]
\NormalTok{bird\_ids }\OtherTok{\textless{}{-}} \FunctionTok{unique}\NormalTok{(vulture\_dat}\SpecialCharTok{$}\NormalTok{individual.local.identifier)}

\NormalTok{start\_stop\_table }\OtherTok{\textless{}{-}} \FunctionTok{data.frame}\NormalTok{()}

\ControlFlowTok{for}\NormalTok{(i }\ControlFlowTok{in} \DecValTok{1}\SpecialCharTok{:}\FunctionTok{length}\NormalTok{(bird\_ids))\{}
\NormalTok{  current\_bird }\OtherTok{\textless{}{-}}\NormalTok{ bird\_ids[i]}
\NormalTok{  curr\_records }\OtherTok{\textless{}{-}}\NormalTok{ vulture\_dat[vulture\_dat}\SpecialCharTok{$}\NormalTok{individual.local.identifier }\SpecialCharTok{==}\NormalTok{ current\_bird,]}
\NormalTok{  start\_rec }\OtherTok{\textless{}{-}} \FunctionTok{head}\NormalTok{(curr\_records, }\AttributeTok{n=}\DecValTok{1}\NormalTok{)}\SpecialCharTok{$}\NormalTok{timestamp}
\NormalTok{  stop\_rec }\OtherTok{\textless{}{-}} \FunctionTok{tail}\NormalTok{(curr\_records, }\AttributeTok{n=}\DecValTok{1}\NormalTok{)}\SpecialCharTok{$}\NormalTok{timestamp}
\NormalTok{  n\_days }\OtherTok{\textless{}{-}} \FunctionTok{round}\NormalTok{(}\FunctionTok{as.numeric}\NormalTok{(}\FunctionTok{difftime}\NormalTok{(}\FunctionTok{ymd\_hms}\NormalTok{(stop\_rec), }\FunctionTok{ymd\_hms}\NormalTok{(start\_rec),}\AttributeTok{units =} \StringTok{"days"}\NormalTok{)),}\DecValTok{3}\NormalTok{)}
\NormalTok{  new\_row }\OtherTok{\textless{}{-}} \FunctionTok{c}\NormalTok{(current\_bird,start\_rec, stop\_rec, n\_days)}
  
\NormalTok{  start\_stop\_table }\OtherTok{\textless{}{-}} \FunctionTok{rbind}\NormalTok{(start\_stop\_table, new\_row)}
\NormalTok{\}}

\FunctionTok{colnames}\NormalTok{(start\_stop\_table) }\OtherTok{\textless{}{-}}  \FunctionTok{c}\NormalTok{(}\StringTok{"bird id"}\NormalTok{,}\StringTok{"start"}\NormalTok{,}\StringTok{"stop"}\NormalTok{,}\StringTok{"n days"}\NormalTok{)}

\NormalTok{(start\_stop\_table)}
\end{Highlighting}
\end{Shaded}

\begin{verbatim}
##    bird id                   start                    stop  n days
## 1        8 2013-06-18 23:26:49.000 2014-06-18 22:02:41.000 364.942
## 2       12 2013-06-20 16:49:51.000 2015-09-01 04:52:41.000 802.502
## 3       22 2013-06-20 17:37:38.000 2015-09-01 03:07:07.000 802.395
## 4       47 2013-07-03 16:06:48.000 2015-09-01 04:31:54.000 789.517
## 5       48 2013-07-03 19:10:03.000 2014-09-22 16:35:20.000 445.893
## 6       57 2013-07-05 16:43:23.000 2014-03-21 19:44:56.000 259.126
## 7      123 2014-04-23 19:44:16.000 2015-09-01 04:48:56.000 495.378
## 8       92 2013-08-02 12:47:47.000 2015-09-01 04:21:36.000 759.648
## 9      108 2014-04-21 18:40:25.000 2015-09-01 04:20:15.000 497.403
## 10     126 2014-05-01 13:38:20.000 2015-09-01 04:54:57.000 487.637
## 11       1 2013-06-26 14:07:33.000 2015-09-01 04:28:52.000 796.598
## 12       3 2013-06-26 14:38:58.000 2015-09-01 04:52:35.000 796.593
## 13       6 2013-06-17 13:19:20.000 2015-09-01 04:37:25.000 805.638
## 14      13 2013-06-27 18:49:55.000 2013-12-21 15:16:23.000 176.852
## 15      60 2013-07-05 12:52:58.000 2015-09-01 04:58:46.000 787.671
## 16      75 2013-07-09 15:31:45.000 2015-09-01 04:46:12.000 783.552
## 17      90 2013-07-30 14:34:17.000 2015-09-01 02:56:53.000 762.516
## 18      91 2013-07-31 15:52:51.000 2015-09-01 04:55:41.000 761.544
\end{verbatim}

\begin{enumerate}
\def\labelenumi{\alph{enumi}.}
\setcounter{enumi}{4}
\tightlist
\item
  Simple plot of the data points for one individual
\end{enumerate}

\begin{Shaded}
\begin{Highlighting}[]
\CommentTok{\#select bird 91}

\NormalTok{id }\OtherTok{\textless{}{-}} \DecValTok{91}

\CommentTok{\#subset whole dataset}

\NormalTok{vulture\_91 }\OtherTok{\textless{}{-}}\NormalTok{ vulture\_dat[vulture\_dat}\SpecialCharTok{$}\NormalTok{individual.local.identifier }\SpecialCharTok{==} \DecValTok{91}\NormalTok{,]}

\FunctionTok{plot}\NormalTok{(vulture\_91}\SpecialCharTok{$}\NormalTok{location.long, vulture\_91}\SpecialCharTok{$}\NormalTok{location.lat)}
\end{Highlighting}
\end{Shaded}

\includegraphics{initial_analysis_files/figure-latex/plot the data for 91-1.pdf}

\begin{Shaded}
\begin{Highlighting}[]
\FunctionTok{ggplot}\NormalTok{(vulture\_91, }\FunctionTok{aes}\NormalTok{(}\AttributeTok{x=}\NormalTok{location.long, }\AttributeTok{y=}\NormalTok{location.lat))}\SpecialCharTok{+} \FunctionTok{geom\_point}\NormalTok{()}
\end{Highlighting}
\end{Shaded}

\includegraphics{initial_analysis_files/figure-latex/plot the data for 91-2.pdf}

\begin{enumerate}
\def\labelenumi{\arabic{enumi}.}
\setcounter{enumi}{1}
\item
  Estimate home range for one individual using three methods of your own
  choice.
\item
  Choose Individual and generate Tracks
\end{enumerate}

\begin{Shaded}
\begin{Highlighting}[]
\NormalTok{vulture\_91 }\OtherTok{\textless{}{-}}\NormalTok{ vulture\_dat }\SpecialCharTok{\%\textgreater{}\%}
  \FunctionTok{filter}\NormalTok{(individual.local.identifier }\SpecialCharTok{==} \DecValTok{91}\NormalTok{)}

\FunctionTok{head}\NormalTok{(vulture\_91)}
\end{Highlighting}
\end{Shaded}

\begin{verbatim}
##     event.id visible               timestamp location.long location.lat
## 1 3383407128    true 2013-07-31 15:52:51.000     -81.64166     33.16277
## 2 3383365434    true 2013-07-31 15:54:06.000     -81.64163     33.16284
## 3 3383282583    true 2013-07-31 15:55:20.000     -81.64163     33.16282
## 4 3383407129    true 2013-07-31 15:56:28.000     -81.64168     33.16279
## 5 3383282584    true 2013-07-31 15:58:21.000     -81.64163     33.16279
## 6 3383282585    true 2013-07-31 16:00:19.000     -81.64157     33.16281
##   gps.hdop gps.satellite.count gps.vdop ground.speed heading
## 1      0.9                   9      1.4            0       0
## 2      0.9                   9      1.4            0       0
## 3      0.9                   9      1.4            0       0
## 4      0.9                   9      1.4            0      58
## 5      0.9                   9      1.4            0       0
## 6      0.9                   9      1.4            0     236
##   height.above.ellipsoid manually.marked.outlier sensor.type
## 1                     53                      NA         gps
## 2                     66                      NA         gps
## 3                     59                      NA         gps
## 4                     74                      NA         gps
## 5                     77                      NA         gps
## 6                     75                      NA         gps
##   individual.taxon.canonical.name tag.local.identifier
## 1                  Cathartes aura                  179
## 2                  Cathartes aura                  179
## 3                  Cathartes aura                  179
## 4                  Cathartes aura                  179
## 5                  Cathartes aura                  179
## 6                  Cathartes aura                  179
##   individual.local.identifier
## 1                          91
## 2                          91
## 3                          91
## 4                          91
## 5                          91
## 6                          91
##                                            study.name
## 1 Black Vultures and Turkey Vultures Southeastern USA
## 2 Black Vultures and Turkey Vultures Southeastern USA
## 3 Black Vultures and Turkey Vultures Southeastern USA
## 4 Black Vultures and Turkey Vultures Southeastern USA
## 5 Black Vultures and Turkey Vultures Southeastern USA
## 6 Black Vultures and Turkey Vultures Southeastern USA
\end{verbatim}

\begin{Shaded}
\begin{Highlighting}[]
\CommentTok{\# check timestamp}
\FunctionTok{class}\NormalTok{(vulture\_91}\SpecialCharTok{$}\NormalTok{timestamp)}
\end{Highlighting}
\end{Shaded}

\begin{verbatim}
## [1] "character"
\end{verbatim}

\begin{Shaded}
\begin{Highlighting}[]
\CommentTok{\# convert timestamp to posixct format}
\NormalTok{vulture\_91}\SpecialCharTok{$}\NormalTok{timestamp }\OtherTok{\textless{}{-}} \FunctionTok{ymd\_hms}\NormalTok{(vulture\_91}\SpecialCharTok{$}\NormalTok{timestamp, }\AttributeTok{tz =} \StringTok{"UTC"}\NormalTok{)}
\FunctionTok{head}\NormalTok{(vulture\_91}\SpecialCharTok{$}\NormalTok{timestamp)}
\end{Highlighting}
\end{Shaded}

\begin{verbatim}
## [1] "2013-07-31 15:52:51 UTC" "2013-07-31 15:54:06 UTC"
## [3] "2013-07-31 15:55:20 UTC" "2013-07-31 15:56:28 UTC"
## [5] "2013-07-31 15:58:21 UTC" "2013-07-31 16:00:19 UTC"
\end{verbatim}

\begin{Shaded}
\begin{Highlighting}[]
\FunctionTok{str}\NormalTok{(vulture\_91}\SpecialCharTok{$}\NormalTok{timestamp)}
\end{Highlighting}
\end{Shaded}

\begin{verbatim}
##  POSIXct[1:187316], format: "2013-07-31 15:52:51" "2013-07-31 15:54:06" "2013-07-31 15:55:20" ...
\end{verbatim}

\begin{Shaded}
\begin{Highlighting}[]
\CommentTok{\# make track for vulture 91}
\NormalTok{trk\_91 }\OtherTok{\textless{}{-}} \FunctionTok{make\_track}\NormalTok{(vulture\_91, location.long,location.lat, timestamp, }\AttributeTok{id =}\NormalTok{ individual.local.identifier, }\AttributeTok{crs =} \DecValTok{4326}\NormalTok{)}

\CommentTok{\# save this file to disk}
\FunctionTok{saveRDS}\NormalTok{(trk\_91, }\AttributeTok{file =} \StringTok{"../output/vulture\_91\_gps\_data\_track.rds"}\NormalTok{)}

\CommentTok{\# check sampling rate for bird 91}
\FunctionTok{summarize\_sampling\_rate}\NormalTok{(trk\_91)}
\end{Highlighting}
\end{Shaded}

\begin{verbatim}
## # A tibble: 1 x 9
##     min    q1 median  mean    q3   max    sd      n unit 
##   <dbl> <dbl>  <dbl> <dbl> <dbl> <dbl> <dbl>  <int> <chr>
## 1  0.55     1   1.03  5.85  1.98 1492.  14.4 187315 min
\end{verbatim}

\begin{Shaded}
\begin{Highlighting}[]
\CommentTok{\# This suggests that the median sampling rate is 2h, but varying up to}
\CommentTok{\# 12h. We can now resample the whole track to 2h interval (with tolerance of}
\CommentTok{\# 10 min), so that if there are more than 2h between relocations, they will }
\CommentTok{\# be divided into different bursts.}

\NormalTok{trk\_91\_resamp }\OtherTok{\textless{}{-}} \FunctionTok{track\_resample}\NormalTok{(trk\_91, }\AttributeTok{rate =} \FunctionTok{minutes}\NormalTok{(}\DecValTok{1}\NormalTok{), }\AttributeTok{tolerance =} \FunctionTok{minutes}\NormalTok{(}\DecValTok{1490}\NormalTok{))}

\CommentTok{\# add step length as a new col}
\NormalTok{trk\_91\_sl }\OtherTok{\textless{}{-}}\NormalTok{ trk\_91\_resamp }\SpecialCharTok{\%\textgreater{}\%} \FunctionTok{mutate}\NormalTok{(}\AttributeTok{sl =} \FunctionTok{step\_lengths}\NormalTok{(.)) }

\CommentTok{\# calculate steps by burst}
\NormalTok{trk\_91\_sbb }\OtherTok{\textless{}{-}}\NormalTok{ trk\_91\_resamp }\SpecialCharTok{\%\textgreater{}\%} \FunctionTok{steps\_by\_burst}\NormalTok{()}
\end{Highlighting}
\end{Shaded}

\begin{enumerate}
\def\labelenumi{\alph{enumi}.}
\tightlist
\item
  MCP Home Range
\end{enumerate}

\begin{Shaded}
\begin{Highlighting}[]
\NormalTok{mcps }\OtherTok{\textless{}{-}} \FunctionTok{hr\_mcp}\NormalTok{(trk\_91, }\AttributeTok{levels =} \FunctionTok{c}\NormalTok{(}\FloatTok{0.5}\NormalTok{, }\FloatTok{0.75}\NormalTok{, }\FloatTok{0.95}\NormalTok{, }\DecValTok{1}\NormalTok{))}
\NormalTok{mcps}
\end{Highlighting}
\end{Shaded}

\begin{verbatim}
## $mcp
## Simple feature collection with 4 features and 3 fields
## Geometry type: POLYGON
## Dimension:     XY
## Bounding box:  xmin: -81.81408 ymin: 32.77979 xmax: -81.16542 ymax: 33.9309
## Geodetic CRS:  WGS 84
##   level     what             area                       geometry
## 1  0.50 estimate  440882403 [m^2] POLYGON ((-81.39618 33.1804...
## 2  0.75 estimate 1401770831 [m^2] POLYGON ((-81.29887 33.2017...
## 3  0.95 estimate 5006729740 [m^2] POLYGON ((-81.17934 33.1347...
## 4  1.00 estimate 5983974085 [m^2] POLYGON ((-81.17934 33.1347...
## 
## $levels
## [1] 0.50 0.75 0.95 1.00
## 
## $estimator
## [1] "mcp"
## 
## $crs
## Coordinate Reference System:
##   User input: EPSG:4326 
##   wkt:
## GEOGCRS["WGS 84",
##     ENSEMBLE["World Geodetic System 1984 ensemble",
##         MEMBER["World Geodetic System 1984 (Transit)"],
##         MEMBER["World Geodetic System 1984 (G730)"],
##         MEMBER["World Geodetic System 1984 (G873)"],
##         MEMBER["World Geodetic System 1984 (G1150)"],
##         MEMBER["World Geodetic System 1984 (G1674)"],
##         MEMBER["World Geodetic System 1984 (G1762)"],
##         MEMBER["World Geodetic System 1984 (G2139)"],
##         ELLIPSOID["WGS 84",6378137,298.257223563,
##             LENGTHUNIT["metre",1]],
##         ENSEMBLEACCURACY[2.0]],
##     PRIMEM["Greenwich",0,
##         ANGLEUNIT["degree",0.0174532925199433]],
##     CS[ellipsoidal,2],
##         AXIS["geodetic latitude (Lat)",north,
##             ORDER[1],
##             ANGLEUNIT["degree",0.0174532925199433]],
##         AXIS["geodetic longitude (Lon)",east,
##             ORDER[2],
##             ANGLEUNIT["degree",0.0174532925199433]],
##     USAGE[
##         SCOPE["Horizontal component of 3D system."],
##         AREA["World."],
##         BBOX[-90,-180,90,180]],
##     ID["EPSG",4326]]
## 
## $data
## # A tibble: 187,316 x 4
##       x_    y_ t_                     id
##  * <dbl> <dbl> <dttm>              <int>
##  1 -81.6  33.2 2013-07-31 15:52:51    91
##  2 -81.6  33.2 2013-07-31 15:54:06    91
##  3 -81.6  33.2 2013-07-31 15:55:20    91
##  4 -81.6  33.2 2013-07-31 15:56:28    91
##  5 -81.6  33.2 2013-07-31 15:58:21    91
##  6 -81.6  33.2 2013-07-31 16:00:19    91
##  7 -81.6  33.2 2013-07-31 16:02:13    91
##  8 -81.6  33.2 2013-07-31 16:04:07    91
##  9 -81.6  33.2 2013-07-31 16:06:50    91
## 10 -81.6  33.2 2013-07-31 16:08:45    91
## # i 187,306 more rows
## 
## attr(,"class")
## [1] "mcp"     "hr_geom" "hr"
\end{verbatim}

\begin{Shaded}
\begin{Highlighting}[]
\CommentTok{\# Get area in km\^{}2}
\FunctionTok{hr\_area}\NormalTok{(mcps, }\AttributeTok{units =} \ConstantTok{TRUE}\NormalTok{) }\SpecialCharTok{\%\textgreater{}\%} 
  \FunctionTok{mutate}\NormalTok{(}\AttributeTok{area =}\NormalTok{ units}\SpecialCharTok{::}\FunctionTok{set\_units}\NormalTok{(area, }\StringTok{"km\^{}2"}\NormalTok{))}
\end{Highlighting}
\end{Shaded}

\begin{verbatim}
## # A tibble: 4 x 3
##   level what       area
##   <dbl> <chr>    [km^2]
## 1  1    estimate  5984.
## 2  0.95 estimate  5007.
## 3  0.75 estimate  1402.
## 4  0.5  estimate   441.
\end{verbatim}

\begin{Shaded}
\begin{Highlighting}[]
\CommentTok{\# plot the polygon isopleths by themeself}
\FunctionTok{plot}\NormalTok{(}\FunctionTok{hr\_isopleths}\NormalTok{(mcps)[}\DecValTok{1}\NormalTok{])}
\end{Highlighting}
\end{Shaded}

\includegraphics{initial_analysis_files/figure-latex/mcp home range estimation for 91-1.pdf}

\begin{Shaded}
\begin{Highlighting}[]
\CommentTok{\# Plot the isopleth with data on top}
\FunctionTok{plot}\NormalTok{(mcps) }
\end{Highlighting}
\end{Shaded}

\includegraphics{initial_analysis_files/figure-latex/mcp home range estimation for 91-2.pdf}

\begin{Shaded}
\begin{Highlighting}[]
\CommentTok{\# Custom plot with ggplot2 (thanks to Brian Smith for this code chunk)}

\FunctionTok{hr\_isopleths}\NormalTok{(mcps) }\SpecialCharTok{\%\textgreater{}\%} 
  \CommentTok{\# Make level a factor for discrete color scales}
  \CommentTok{\# Can control order and labels here}
  \FunctionTok{mutate}\NormalTok{(}\AttributeTok{level =} \FunctionTok{factor}\NormalTok{(level, }
                        \AttributeTok{levels =} \FunctionTok{c}\NormalTok{(}\StringTok{"1"}\NormalTok{, }\StringTok{"0.95"}\NormalTok{, }\StringTok{"0.75"}\NormalTok{, }\StringTok{"0.5"}\NormalTok{),}
                        \AttributeTok{labels =} \FunctionTok{c}\NormalTok{(}\StringTok{"100\%"}\NormalTok{, }\StringTok{"95\%"}\NormalTok{, }\StringTok{"75\%"}\NormalTok{, }\StringTok{"50\%"}\NormalTok{))) }\SpecialCharTok{\%\textgreater{}\%} 
  \FunctionTok{ggplot}\NormalTok{() }\SpecialCharTok{+}
  \FunctionTok{geom\_sf}\NormalTok{(}\FunctionTok{aes}\NormalTok{(}\AttributeTok{color =}\NormalTok{ level), }
          \AttributeTok{fill =} \ConstantTok{NA}\NormalTok{, }\AttributeTok{linewidth =} \DecValTok{2}\NormalTok{) }\SpecialCharTok{+}
  \FunctionTok{geom\_point}\NormalTok{(}\AttributeTok{data =}\NormalTok{ mcps}\SpecialCharTok{$}\NormalTok{data, }\FunctionTok{aes}\NormalTok{(}\AttributeTok{x =}\NormalTok{ x\_, }\AttributeTok{y =}\NormalTok{ y\_),}
             \AttributeTok{color =} \StringTok{"gray30"}\NormalTok{, }\AttributeTok{alpha =} \FloatTok{0.5}\NormalTok{) }\SpecialCharTok{+}
  \FunctionTok{xlab}\NormalTok{(}\ConstantTok{NULL}\NormalTok{) }\SpecialCharTok{+}
  \FunctionTok{ylab}\NormalTok{(}\ConstantTok{NULL}\NormalTok{) }\SpecialCharTok{+}
  \FunctionTok{scale\_color\_brewer}\NormalTok{(}\AttributeTok{name =} \StringTok{"MCP Level"}\NormalTok{,}
                     \AttributeTok{palette =} \StringTok{"Set1"}\NormalTok{) }\SpecialCharTok{+}
  \FunctionTok{theme\_bw}\NormalTok{() }\SpecialCharTok{+}
  \FunctionTok{theme}\NormalTok{(}\AttributeTok{legend.position =} \StringTok{"bottom"}\NormalTok{, }
        \AttributeTok{legend.box.background =} \FunctionTok{element\_rect}\NormalTok{(}\AttributeTok{colour =} \StringTok{"black"}\NormalTok{, }\AttributeTok{linewidth =} \FloatTok{0.8}\NormalTok{))}
\end{Highlighting}
\end{Shaded}

\includegraphics{initial_analysis_files/figure-latex/mcp home range estimation for 91-3.pdf}

\begin{enumerate}
\def\labelenumi{\alph{enumi}.}
\setcounter{enumi}{1}
\tightlist
\item
  Kernel density estimation of Home Range
\end{enumerate}

\begin{Shaded}
\begin{Highlighting}[]
\NormalTok{kdes }\OtherTok{\textless{}{-}} \FunctionTok{hr\_kde}\NormalTok{(trk\_91, }\AttributeTok{levels =} \FunctionTok{c}\NormalTok{(}\FloatTok{0.5}\NormalTok{, }\FloatTok{0.95}\NormalTok{),)}

\CommentTok{\# Get area in km\^{}2}
\FunctionTok{hr\_area}\NormalTok{(kdes, }\AttributeTok{units =} \ConstantTok{TRUE}\NormalTok{) }\SpecialCharTok{\%\textgreater{}\%} 
  \FunctionTok{mutate}\NormalTok{(}\AttributeTok{area =}\NormalTok{ units}\SpecialCharTok{::}\FunctionTok{set\_units}\NormalTok{(area, }\StringTok{"km\^{}2"}\NormalTok{))}
\end{Highlighting}
\end{Shaded}

\begin{verbatim}
## # A tibble: 2 x 3
##   level what       area
##   <dbl> <chr>    [km^2]
## 1  0.95 estimate  2689.
## 2  0.5  estimate   169.
\end{verbatim}

\begin{Shaded}
\begin{Highlighting}[]
\CommentTok{\# plot the polygon isopleths by themeself}
\FunctionTok{plot}\NormalTok{(}\FunctionTok{hr\_isopleths}\NormalTok{(kdes)[}\DecValTok{1}\NormalTok{])}
\end{Highlighting}
\end{Shaded}

\includegraphics{initial_analysis_files/figure-latex/KDE home range estimation for 91-1.pdf}

\begin{Shaded}
\begin{Highlighting}[]
\CommentTok{\# Plot the isopleth with data on top}
\FunctionTok{plot}\NormalTok{(kdes) }
\end{Highlighting}
\end{Shaded}

\includegraphics{initial_analysis_files/figure-latex/KDE home range estimation for 91-2.pdf}

\begin{Shaded}
\begin{Highlighting}[]
\CommentTok{\# Custom plot with ggplot2 (thanks to Brian Smith for this code chunk)}

\FunctionTok{hr\_isopleths}\NormalTok{(kdes) }\SpecialCharTok{\%\textgreater{}\%} 
  \CommentTok{\# Make level a factor for discrete color scales}
  \CommentTok{\# Can control order and labels here}
  \FunctionTok{mutate}\NormalTok{(}\AttributeTok{level =} \FunctionTok{factor}\NormalTok{(level, }
                        \AttributeTok{levels =} \FunctionTok{c}\NormalTok{(}\StringTok{"1"}\NormalTok{, }\StringTok{"0.95"}\NormalTok{, }\StringTok{"0.75"}\NormalTok{, }\StringTok{"0.5"}\NormalTok{),}
                        \AttributeTok{labels =} \FunctionTok{c}\NormalTok{(}\StringTok{"100\%"}\NormalTok{, }\StringTok{"95\%"}\NormalTok{, }\StringTok{"75\%"}\NormalTok{, }\StringTok{"50\%"}\NormalTok{))) }\SpecialCharTok{\%\textgreater{}\%} 
  \FunctionTok{ggplot}\NormalTok{() }\SpecialCharTok{+}
  \FunctionTok{geom\_sf}\NormalTok{(}\FunctionTok{aes}\NormalTok{(}\AttributeTok{color =}\NormalTok{ level), }
          \AttributeTok{fill =} \ConstantTok{NA}\NormalTok{, }\AttributeTok{linewidth =} \DecValTok{2}\NormalTok{) }\SpecialCharTok{+}
  \FunctionTok{geom\_point}\NormalTok{(}\AttributeTok{data =}\NormalTok{ mcps}\SpecialCharTok{$}\NormalTok{data, }\FunctionTok{aes}\NormalTok{(}\AttributeTok{x =}\NormalTok{ x\_, }\AttributeTok{y =}\NormalTok{ y\_),}
             \AttributeTok{color =} \StringTok{"gray30"}\NormalTok{, }\AttributeTok{alpha =} \FloatTok{0.5}\NormalTok{) }\SpecialCharTok{+}
  \FunctionTok{xlab}\NormalTok{(}\ConstantTok{NULL}\NormalTok{) }\SpecialCharTok{+}
  \FunctionTok{ylab}\NormalTok{(}\ConstantTok{NULL}\NormalTok{) }\SpecialCharTok{+}
  \FunctionTok{scale\_color\_brewer}\NormalTok{(}\AttributeTok{name =} \StringTok{"MCP Level"}\NormalTok{,}
                     \AttributeTok{palette =} \StringTok{"Set1"}\NormalTok{) }\SpecialCharTok{+}
  \FunctionTok{theme\_bw}\NormalTok{() }\SpecialCharTok{+}
  \FunctionTok{theme}\NormalTok{(}\AttributeTok{legend.position =} \StringTok{"bottom"}\NormalTok{, }
        \AttributeTok{legend.box.background =} \FunctionTok{element\_rect}\NormalTok{(}\AttributeTok{colour =} \StringTok{"black"}\NormalTok{, }\AttributeTok{linewidth =} \FloatTok{0.8}\NormalTok{))}
\end{Highlighting}
\end{Shaded}

\includegraphics{initial_analysis_files/figure-latex/KDE home range estimation for 91-3.pdf}

\begin{Shaded}
\begin{Highlighting}[]
\FunctionTok{hr\_overlap}\NormalTok{(mcps,kdes)}
\end{Highlighting}
\end{Shaded}

\begin{verbatim}
## # A tibble: 4 x 2
##   levels overlap
##    <dbl>   <dbl>
## 1   1     0.449 
## 2   0.95  0.0337
## 3   0.75  0     
## 4   0.5   0
\end{verbatim}

\begin{enumerate}
\def\labelenumi{\alph{enumi}.}
\setcounter{enumi}{2}
\tightlist
\item
  Autocorrelated Kernel Density estimator of home range
\end{enumerate}

\begin{Shaded}
\begin{Highlighting}[]
\CommentTok{\# Without going into detail on the different CTMMs, we\textquotesingle{}ll demonstrate fitting}
\CommentTok{\# an aKDE with an Ornstein{-}Uhlenbeck model.}
\NormalTok{akdes }\OtherTok{\textless{}{-}} \FunctionTok{hr\_akde}\NormalTok{(trk\_91, }\AttributeTok{model =} \FunctionTok{fit\_ctmm}\NormalTok{(trk\_91, }\StringTok{"iid"}\NormalTok{), }\AttributeTok{levels =} \FloatTok{0.95}\NormalTok{)}

\CommentTok{\# Get area in km\^{}2}
\FunctionTok{hr\_area}\NormalTok{(akdes, }\AttributeTok{units =} \ConstantTok{TRUE}\NormalTok{) }\SpecialCharTok{\%\textgreater{}\%} 
  \FunctionTok{mutate}\NormalTok{(}\AttributeTok{area =}\NormalTok{ units}\SpecialCharTok{::}\FunctionTok{set\_units}\NormalTok{(area, }\StringTok{"km\^{}2"}\NormalTok{))}
\end{Highlighting}
\end{Shaded}

\begin{verbatim}
## # A tibble: 3 x 3
##   level what         area
##   <dbl> <chr>      [km^2]
## 1  0.95 lci (0.95)  2640.
## 2  0.95 estimate    2651.
## 3  0.95 uci (0.95)  2662.
\end{verbatim}

\begin{Shaded}
\begin{Highlighting}[]
\CommentTok{\# plot the polygon isopleths by themeself}
\FunctionTok{plot}\NormalTok{(}\FunctionTok{hr\_isopleths}\NormalTok{(akdes)[}\DecValTok{1}\NormalTok{])}
\end{Highlighting}
\end{Shaded}

\includegraphics{initial_analysis_files/figure-latex/unnamed-chunk-1-1.pdf}

\begin{Shaded}
\begin{Highlighting}[]
\CommentTok{\# Plot the isopleth with data on top}
\FunctionTok{plot}\NormalTok{(akdes) }
\end{Highlighting}
\end{Shaded}

\includegraphics{initial_analysis_files/figure-latex/unnamed-chunk-1-2.pdf}

\begin{Shaded}
\begin{Highlighting}[]
\CommentTok{\# Custom plot with ggplot2 (thanks to Brian Smith for this code chunk)}

\FunctionTok{hr\_isopleths}\NormalTok{(akdes) }\SpecialCharTok{\%\textgreater{}\%} 
  \CommentTok{\# Make level a factor for discrete color scales}
  \CommentTok{\# Can control order and labels here}
  \FunctionTok{mutate}\NormalTok{(}\AttributeTok{level =} \FunctionTok{factor}\NormalTok{(level, }
                        \AttributeTok{levels =} \FunctionTok{c}\NormalTok{(}\StringTok{"1"}\NormalTok{, }\StringTok{"0.95"}\NormalTok{, }\StringTok{"0.75"}\NormalTok{, }\StringTok{"0.5"}\NormalTok{),}
                        \AttributeTok{labels =} \FunctionTok{c}\NormalTok{(}\StringTok{"100\%"}\NormalTok{, }\StringTok{"95\%"}\NormalTok{, }\StringTok{"75\%"}\NormalTok{, }\StringTok{"50\%"}\NormalTok{))) }\SpecialCharTok{\%\textgreater{}\%} 
  \FunctionTok{ggplot}\NormalTok{() }\SpecialCharTok{+}
  \FunctionTok{geom\_sf}\NormalTok{(}\FunctionTok{aes}\NormalTok{(}\AttributeTok{color =}\NormalTok{ level), }
          \AttributeTok{fill =} \ConstantTok{NA}\NormalTok{, }\AttributeTok{linewidth =} \DecValTok{2}\NormalTok{) }\SpecialCharTok{+}
  \FunctionTok{geom\_point}\NormalTok{(}\AttributeTok{data =}\NormalTok{ mcps}\SpecialCharTok{$}\NormalTok{data, }\FunctionTok{aes}\NormalTok{(}\AttributeTok{x =}\NormalTok{ x\_, }\AttributeTok{y =}\NormalTok{ y\_),}
             \AttributeTok{color =} \StringTok{"gray30"}\NormalTok{, }\AttributeTok{alpha =} \FloatTok{0.5}\NormalTok{) }\SpecialCharTok{+}
  \FunctionTok{xlab}\NormalTok{(}\ConstantTok{NULL}\NormalTok{) }\SpecialCharTok{+}
  \FunctionTok{ylab}\NormalTok{(}\ConstantTok{NULL}\NormalTok{) }\SpecialCharTok{+}
  \FunctionTok{scale\_color\_brewer}\NormalTok{(}\AttributeTok{name =} \StringTok{"MCP Level"}\NormalTok{,}
                     \AttributeTok{palette =} \StringTok{"Set1"}\NormalTok{) }\SpecialCharTok{+}
  \FunctionTok{theme\_bw}\NormalTok{() }\SpecialCharTok{+}
  \FunctionTok{theme}\NormalTok{(}\AttributeTok{legend.position =} \StringTok{"bottom"}\NormalTok{, }
        \AttributeTok{legend.box.background =} \FunctionTok{element\_rect}\NormalTok{(}\AttributeTok{colour =} \StringTok{"black"}\NormalTok{, }\AttributeTok{linewidth =} \FloatTok{0.8}\NormalTok{))}
\end{Highlighting}
\end{Shaded}

\includegraphics{initial_analysis_files/figure-latex/unnamed-chunk-1-3.pdf}

\begin{Shaded}
\begin{Highlighting}[]
\FunctionTok{hr\_overlap}\NormalTok{(kdes,akdes)}
\end{Highlighting}
\end{Shaded}

\begin{verbatim}
## # A tibble: 2 x 2
##   levels overlap
##    <dbl>   <dbl>
## 1   0.95   0.959
## 2   0.5    1
\end{verbatim}

\hypertarget{examine-bird-108}{%
\subsubsection{Examine Bird 108}\label{examine-bird-108}}

\begin{enumerate}
\def\labelenumi{\alph{enumi}.}
\setcounter{enumi}{4}
\tightlist
\item
  Simple plot of the data points for one individual
\end{enumerate}

\begin{Shaded}
\begin{Highlighting}[]
\CommentTok{\#select bird 108}

\NormalTok{id }\OtherTok{\textless{}{-}} \DecValTok{108}

\CommentTok{\#subset whole dataset}

\NormalTok{vulture\_108 }\OtherTok{\textless{}{-}}\NormalTok{ vulture\_dat[vulture\_dat}\SpecialCharTok{$}\NormalTok{individual.local.identifier }\SpecialCharTok{==} \DecValTok{108}\NormalTok{,]}

\FunctionTok{plot}\NormalTok{(vulture\_108}\SpecialCharTok{$}\NormalTok{location.long, vulture\_108}\SpecialCharTok{$}\NormalTok{location.lat)}
\end{Highlighting}
\end{Shaded}

\includegraphics{initial_analysis_files/figure-latex/plot the data-1.pdf}

\begin{Shaded}
\begin{Highlighting}[]
\FunctionTok{ggplot}\NormalTok{(vulture\_108, }\FunctionTok{aes}\NormalTok{(}\AttributeTok{x=}\NormalTok{location.long, }\AttributeTok{y=}\NormalTok{location.lat))}\SpecialCharTok{+} \FunctionTok{geom\_point}\NormalTok{()}
\end{Highlighting}
\end{Shaded}

\includegraphics{initial_analysis_files/figure-latex/plot the data-2.pdf}

\begin{enumerate}
\def\labelenumi{\arabic{enumi}.}
\setcounter{enumi}{1}
\item
  Estimate home range for one individual using three methods of your own
  choice.
\item
  Choose Individual and generate Tracks
\end{enumerate}

\begin{Shaded}
\begin{Highlighting}[]
\NormalTok{vulture\_108 }\OtherTok{\textless{}{-}}\NormalTok{ vulture\_dat }\SpecialCharTok{\%\textgreater{}\%}
  \FunctionTok{filter}\NormalTok{(individual.local.identifier }\SpecialCharTok{==} \DecValTok{108}\NormalTok{)}

\FunctionTok{head}\NormalTok{(vulture\_108)}
\end{Highlighting}
\end{Shaded}

\begin{verbatim}
##     event.id visible               timestamp location.long location.lat
## 1 3381246783    true 2014-04-21 18:40:25.000     -81.73844     33.26194
## 2 3381246784    true 2014-04-21 18:52:55.000     -81.73833     33.26218
## 3 3381246785    true 2014-04-21 19:05:17.000     -81.73848     33.26196
## 4 3381246786    true 2014-04-21 19:18:03.000     -81.73826     33.26196
## 5 3381246787    true 2014-04-21 19:31:25.000     -81.73846     33.26190
## 6 3381246788    true 2014-04-21 19:44:51.000     -81.73858     33.26191
##   gps.hdop gps.satellite.count gps.vdop ground.speed heading
## 1      1.7                   6      2.4            0       0
## 2      1.9                   5      2.4            0     263
## 3      1.6                   6      2.2            0       0
## 4      2.1                   6      3.3            0       0
## 5      2.9                   6      4.9            0       0
## 6      5.6                   6      9.6            0      59
##   height.above.ellipsoid manually.marked.outlier sensor.type
## 1                     29                      NA         gps
## 2                     60                      NA         gps
## 3                     63                      NA         gps
## 4                     61                      NA         gps
## 5                     77                      NA         gps
## 6                     77                      NA         gps
##   individual.taxon.canonical.name tag.local.identifier
## 1                Coragyps atratus                  168
## 2                Coragyps atratus                  168
## 3                Coragyps atratus                  168
## 4                Coragyps atratus                  168
## 5                Coragyps atratus                  168
## 6                Coragyps atratus                  168
##   individual.local.identifier
## 1                         108
## 2                         108
## 3                         108
## 4                         108
## 5                         108
## 6                         108
##                                            study.name
## 1 Black Vultures and Turkey Vultures Southeastern USA
## 2 Black Vultures and Turkey Vultures Southeastern USA
## 3 Black Vultures and Turkey Vultures Southeastern USA
## 4 Black Vultures and Turkey Vultures Southeastern USA
## 5 Black Vultures and Turkey Vultures Southeastern USA
## 6 Black Vultures and Turkey Vultures Southeastern USA
\end{verbatim}

\begin{Shaded}
\begin{Highlighting}[]
\CommentTok{\# check timestamp}
\FunctionTok{class}\NormalTok{(vulture\_108}\SpecialCharTok{$}\NormalTok{timestamp)}
\end{Highlighting}
\end{Shaded}

\begin{verbatim}
## [1] "character"
\end{verbatim}

\begin{Shaded}
\begin{Highlighting}[]
\CommentTok{\# convert timestamp to posixct format}
\NormalTok{vulture\_108}\SpecialCharTok{$}\NormalTok{timestamp }\OtherTok{\textless{}{-}} \FunctionTok{ymd\_hms}\NormalTok{(vulture\_108}\SpecialCharTok{$}\NormalTok{timestamp, }\AttributeTok{tz =} \StringTok{"UTC"}\NormalTok{)}
\FunctionTok{head}\NormalTok{(vulture\_108}\SpecialCharTok{$}\NormalTok{timestamp)}
\end{Highlighting}
\end{Shaded}

\begin{verbatim}
## [1] "2014-04-21 18:40:25 UTC" "2014-04-21 18:52:55 UTC"
## [3] "2014-04-21 19:05:17 UTC" "2014-04-21 19:18:03 UTC"
## [5] "2014-04-21 19:31:25 UTC" "2014-04-21 19:44:51 UTC"
\end{verbatim}

\begin{Shaded}
\begin{Highlighting}[]
\FunctionTok{str}\NormalTok{(vulture\_108}\SpecialCharTok{$}\NormalTok{timestamp)}
\end{Highlighting}
\end{Shaded}

\begin{verbatim}
##  POSIXct[1:87646], format: "2014-04-21 18:40:25" "2014-04-21 18:52:55" "2014-04-21 19:05:17" ...
\end{verbatim}

\begin{Shaded}
\begin{Highlighting}[]
\CommentTok{\# make track for vulture 91}
\NormalTok{trk\_108 }\OtherTok{\textless{}{-}} \FunctionTok{make\_track}\NormalTok{(vulture\_108, location.long,location.lat, timestamp, }\AttributeTok{id =}\NormalTok{ individual.local.identifier, }\AttributeTok{crs =} \DecValTok{4326}\NormalTok{)}

\CommentTok{\# save this file to disk}
\FunctionTok{saveRDS}\NormalTok{(trk\_108, }\AttributeTok{file =} \StringTok{"../output/vulture\_108\_gps\_data\_track.rds"}\NormalTok{)}

\CommentTok{\# check sampling rate for bird 108}
\FunctionTok{summarize\_sampling\_rate}\NormalTok{(trk\_108)}
\end{Highlighting}
\end{Shaded}

\begin{verbatim}
## # A tibble: 1 x 9
##     min    q1 median  mean    q3   max    sd     n unit 
##   <dbl> <dbl>  <dbl> <dbl> <dbl> <dbl> <dbl> <int> <chr>
## 1 0.667  1.02   1.77  8.17  6.28  300.  15.8 87645 min
\end{verbatim}

\begin{Shaded}
\begin{Highlighting}[]
\CommentTok{\# This suggests that the median sampling rate is 2h, but varying up to}
\CommentTok{\# 12h. We can now resample the whole track to 2h interval (with tolerance of}
\CommentTok{\# 10 min), so that if there are more than 2h between relocations, they will }
\CommentTok{\# be divided into different bursts.}

\NormalTok{trk\_108\_resamp }\OtherTok{\textless{}{-}} \FunctionTok{track\_resample}\NormalTok{(trk\_108, }\AttributeTok{rate =} \FunctionTok{minutes}\NormalTok{(}\DecValTok{2}\NormalTok{), }\AttributeTok{tolerance =} \FunctionTok{minutes}\NormalTok{(}\DecValTok{300}\NormalTok{))}

\CommentTok{\# add step length as a new col}
\NormalTok{trk\_108\_sl }\OtherTok{\textless{}{-}}\NormalTok{ trk\_108\_resamp }\SpecialCharTok{\%\textgreater{}\%} \FunctionTok{mutate}\NormalTok{(}\AttributeTok{sl =} \FunctionTok{step\_lengths}\NormalTok{(.)) }

\CommentTok{\# calculate steps by burst}
\NormalTok{trk\_108\_sbb }\OtherTok{\textless{}{-}}\NormalTok{ trk\_108\_resamp }\SpecialCharTok{\%\textgreater{}\%} \FunctionTok{steps\_by\_burst}\NormalTok{()}
\end{Highlighting}
\end{Shaded}

\begin{enumerate}
\def\labelenumi{\alph{enumi}.}
\tightlist
\item
  MCP Home Range
\end{enumerate}

\begin{Shaded}
\begin{Highlighting}[]
\NormalTok{mcps }\OtherTok{\textless{}{-}} \FunctionTok{hr\_mcp}\NormalTok{(trk\_108, }\AttributeTok{levels =} \FunctionTok{c}\NormalTok{(}\FloatTok{0.5}\NormalTok{, }\FloatTok{0.75}\NormalTok{, }\FloatTok{0.95}\NormalTok{, }\DecValTok{1}\NormalTok{))}
\NormalTok{mcps}
\end{Highlighting}
\end{Shaded}

\begin{verbatim}
## $mcp
## Simple feature collection with 4 features and 3 fields
## Geometry type: POLYGON
## Dimension:     XY
## Bounding box:  xmin: -82.51917 ymin: 32.52645 xmax: -81.40612 ymax: 33.86881
## Geodetic CRS:  WGS 84
##   level     what             area                       geometry
## 1  0.50 estimate  858545605 [m^2] POLYGON ((-81.8695 33.16665...
## 2  0.75 estimate 2743379417 [m^2] POLYGON ((-81.73823 33.2636...
## 3  0.95 estimate 6871757454 [m^2] POLYGON ((-81.50507 32.8004...
## 4  1.00 estimate 8166896342 [m^2] POLYGON ((-81.40612 32.7443...
## 
## $levels
## [1] 0.50 0.75 0.95 1.00
## 
## $estimator
## [1] "mcp"
## 
## $crs
## Coordinate Reference System:
##   User input: EPSG:4326 
##   wkt:
## GEOGCRS["WGS 84",
##     ENSEMBLE["World Geodetic System 1984 ensemble",
##         MEMBER["World Geodetic System 1984 (Transit)"],
##         MEMBER["World Geodetic System 1984 (G730)"],
##         MEMBER["World Geodetic System 1984 (G873)"],
##         MEMBER["World Geodetic System 1984 (G1150)"],
##         MEMBER["World Geodetic System 1984 (G1674)"],
##         MEMBER["World Geodetic System 1984 (G1762)"],
##         MEMBER["World Geodetic System 1984 (G2139)"],
##         ELLIPSOID["WGS 84",6378137,298.257223563,
##             LENGTHUNIT["metre",1]],
##         ENSEMBLEACCURACY[2.0]],
##     PRIMEM["Greenwich",0,
##         ANGLEUNIT["degree",0.0174532925199433]],
##     CS[ellipsoidal,2],
##         AXIS["geodetic latitude (Lat)",north,
##             ORDER[1],
##             ANGLEUNIT["degree",0.0174532925199433]],
##         AXIS["geodetic longitude (Lon)",east,
##             ORDER[2],
##             ANGLEUNIT["degree",0.0174532925199433]],
##     USAGE[
##         SCOPE["Horizontal component of 3D system."],
##         AREA["World."],
##         BBOX[-90,-180,90,180]],
##     ID["EPSG",4326]]
## 
## $data
## # A tibble: 87,646 x 4
##       x_    y_ t_                            id
##  * <dbl> <dbl> <dttm>                     <int>
##  1 -81.7  33.3 2014-04-21 18:40:25.000000   108
##  2 -81.7  33.3 2014-04-21 18:52:55.000000   108
##  3 -81.7  33.3 2014-04-21 19:05:17.000000   108
##  4 -81.7  33.3 2014-04-21 19:18:03.000000   108
##  5 -81.7  33.3 2014-04-21 19:31:25.000000   108
##  6 -81.7  33.3 2014-04-21 19:44:51.000000   108
##  7 -81.7  33.3 2014-04-21 19:53:19.000000   108
##  8 -81.7  33.3 2014-04-21 19:58:48.000000   108
##  9 -81.7  33.3 2014-04-21 20:03:08.000000   108
## 10 -81.7  33.3 2014-04-21 20:09:16.000000   108
## # i 87,636 more rows
## 
## attr(,"class")
## [1] "mcp"     "hr_geom" "hr"
\end{verbatim}

\begin{Shaded}
\begin{Highlighting}[]
\CommentTok{\# Get area in km\^{}2}
\FunctionTok{hr\_area}\NormalTok{(mcps, }\AttributeTok{units =} \ConstantTok{TRUE}\NormalTok{) }\SpecialCharTok{\%\textgreater{}\%} 
  \FunctionTok{mutate}\NormalTok{(}\AttributeTok{area =}\NormalTok{ units}\SpecialCharTok{::}\FunctionTok{set\_units}\NormalTok{(area, }\StringTok{"km\^{}2"}\NormalTok{))}
\end{Highlighting}
\end{Shaded}

\begin{verbatim}
## # A tibble: 4 x 3
##   level what       area
##   <dbl> <chr>    [km^2]
## 1  1    estimate  8167.
## 2  0.95 estimate  6872.
## 3  0.75 estimate  2743.
## 4  0.5  estimate   859.
\end{verbatim}

\begin{Shaded}
\begin{Highlighting}[]
\CommentTok{\# plot the polygon isopleths by themeself}
\FunctionTok{plot}\NormalTok{(}\FunctionTok{hr\_isopleths}\NormalTok{(mcps)[}\DecValTok{1}\NormalTok{])}
\end{Highlighting}
\end{Shaded}

\includegraphics{initial_analysis_files/figure-latex/mcp home range estimation-1.pdf}

\begin{Shaded}
\begin{Highlighting}[]
\CommentTok{\# Plot the isopleth with data on top}
\FunctionTok{plot}\NormalTok{(mcps) }
\end{Highlighting}
\end{Shaded}

\includegraphics{initial_analysis_files/figure-latex/mcp home range estimation-2.pdf}

\begin{Shaded}
\begin{Highlighting}[]
\CommentTok{\# Custom plot with ggplot2 (thanks to Brian Smith for this code chunk)}

\FunctionTok{hr\_isopleths}\NormalTok{(mcps) }\SpecialCharTok{\%\textgreater{}\%} 
  \CommentTok{\# Make level a factor for discrete color scales}
  \CommentTok{\# Can control order and labels here}
  \FunctionTok{mutate}\NormalTok{(}\AttributeTok{level =} \FunctionTok{factor}\NormalTok{(level, }
                        \AttributeTok{levels =} \FunctionTok{c}\NormalTok{(}\StringTok{"1"}\NormalTok{, }\StringTok{"0.95"}\NormalTok{, }\StringTok{"0.75"}\NormalTok{, }\StringTok{"0.5"}\NormalTok{),}
                        \AttributeTok{labels =} \FunctionTok{c}\NormalTok{(}\StringTok{"100\%"}\NormalTok{, }\StringTok{"95\%"}\NormalTok{, }\StringTok{"75\%"}\NormalTok{, }\StringTok{"50\%"}\NormalTok{))) }\SpecialCharTok{\%\textgreater{}\%} 
  \FunctionTok{ggplot}\NormalTok{() }\SpecialCharTok{+}
  \FunctionTok{geom\_sf}\NormalTok{(}\FunctionTok{aes}\NormalTok{(}\AttributeTok{color =}\NormalTok{ level), }
          \AttributeTok{fill =} \ConstantTok{NA}\NormalTok{, }\AttributeTok{linewidth =} \DecValTok{2}\NormalTok{) }\SpecialCharTok{+}
  \FunctionTok{geom\_point}\NormalTok{(}\AttributeTok{data =}\NormalTok{ mcps}\SpecialCharTok{$}\NormalTok{data, }\FunctionTok{aes}\NormalTok{(}\AttributeTok{x =}\NormalTok{ x\_, }\AttributeTok{y =}\NormalTok{ y\_),}
             \AttributeTok{color =} \StringTok{"gray30"}\NormalTok{, }\AttributeTok{alpha =} \FloatTok{0.5}\NormalTok{) }\SpecialCharTok{+}
  \FunctionTok{xlab}\NormalTok{(}\ConstantTok{NULL}\NormalTok{) }\SpecialCharTok{+}
  \FunctionTok{ylab}\NormalTok{(}\ConstantTok{NULL}\NormalTok{) }\SpecialCharTok{+}
  \FunctionTok{scale\_color\_brewer}\NormalTok{(}\AttributeTok{name =} \StringTok{"MCP Level"}\NormalTok{,}
                     \AttributeTok{palette =} \StringTok{"Set1"}\NormalTok{) }\SpecialCharTok{+}
  \FunctionTok{theme\_bw}\NormalTok{() }\SpecialCharTok{+}
  \FunctionTok{theme}\NormalTok{(}\AttributeTok{legend.position =} \StringTok{"bottom"}\NormalTok{, }
        \AttributeTok{legend.box.background =} \FunctionTok{element\_rect}\NormalTok{(}\AttributeTok{colour =} \StringTok{"black"}\NormalTok{, }\AttributeTok{linewidth =} \FloatTok{0.8}\NormalTok{))}
\end{Highlighting}
\end{Shaded}

\includegraphics{initial_analysis_files/figure-latex/mcp home range estimation-3.pdf}

\begin{enumerate}
\def\labelenumi{\alph{enumi}.}
\setcounter{enumi}{1}
\tightlist
\item
  Kernel density estimation of Home Range
\end{enumerate}

\begin{Shaded}
\begin{Highlighting}[]
\NormalTok{kdes }\OtherTok{\textless{}{-}} \FunctionTok{hr\_kde}\NormalTok{(trk\_108, }\AttributeTok{levels =} \FunctionTok{c}\NormalTok{(}\FloatTok{0.5}\NormalTok{, }\FloatTok{0.95}\NormalTok{),)}

\CommentTok{\# Get area in km\^{}2}
\FunctionTok{hr\_area}\NormalTok{(kdes, }\AttributeTok{units =} \ConstantTok{TRUE}\NormalTok{) }\SpecialCharTok{\%\textgreater{}\%} 
  \FunctionTok{mutate}\NormalTok{(}\AttributeTok{area =}\NormalTok{ units}\SpecialCharTok{::}\FunctionTok{set\_units}\NormalTok{(area, }\StringTok{"km\^{}2"}\NormalTok{))}
\end{Highlighting}
\end{Shaded}

\begin{verbatim}
## # A tibble: 2 x 3
##   level what       area
##   <dbl> <chr>    [km^2]
## 1  0.95 estimate  3506.
## 2  0.5  estimate   537.
\end{verbatim}

\begin{Shaded}
\begin{Highlighting}[]
\CommentTok{\# plot the polygon isopleths by themeself}
\FunctionTok{plot}\NormalTok{(}\FunctionTok{hr\_isopleths}\NormalTok{(kdes)[}\DecValTok{1}\NormalTok{])}
\end{Highlighting}
\end{Shaded}

\includegraphics{initial_analysis_files/figure-latex/KDE home range estimation-1.pdf}

\begin{Shaded}
\begin{Highlighting}[]
\CommentTok{\# Plot the isopleth with data on top}
\FunctionTok{plot}\NormalTok{(kdes) }
\end{Highlighting}
\end{Shaded}

\includegraphics{initial_analysis_files/figure-latex/KDE home range estimation-2.pdf}

\begin{Shaded}
\begin{Highlighting}[]
\CommentTok{\# Custom plot with ggplot2 (thanks to Brian Smith for this code chunk)}

\FunctionTok{hr\_isopleths}\NormalTok{(kdes) }\SpecialCharTok{\%\textgreater{}\%} 
  \CommentTok{\# Make level a factor for discrete color scales}
  \CommentTok{\# Can control order and labels here}
  \FunctionTok{mutate}\NormalTok{(}\AttributeTok{level =} \FunctionTok{factor}\NormalTok{(level, }
                        \AttributeTok{levels =} \FunctionTok{c}\NormalTok{(}\StringTok{"1"}\NormalTok{, }\StringTok{"0.95"}\NormalTok{, }\StringTok{"0.75"}\NormalTok{, }\StringTok{"0.5"}\NormalTok{),}
                        \AttributeTok{labels =} \FunctionTok{c}\NormalTok{(}\StringTok{"100\%"}\NormalTok{, }\StringTok{"95\%"}\NormalTok{, }\StringTok{"75\%"}\NormalTok{, }\StringTok{"50\%"}\NormalTok{))) }\SpecialCharTok{\%\textgreater{}\%} 
  \FunctionTok{ggplot}\NormalTok{() }\SpecialCharTok{+}
  \FunctionTok{geom\_sf}\NormalTok{(}\FunctionTok{aes}\NormalTok{(}\AttributeTok{color =}\NormalTok{ level), }
          \AttributeTok{fill =} \ConstantTok{NA}\NormalTok{, }\AttributeTok{linewidth =} \DecValTok{2}\NormalTok{) }\SpecialCharTok{+}
  \FunctionTok{geom\_point}\NormalTok{(}\AttributeTok{data =}\NormalTok{ mcps}\SpecialCharTok{$}\NormalTok{data, }\FunctionTok{aes}\NormalTok{(}\AttributeTok{x =}\NormalTok{ x\_, }\AttributeTok{y =}\NormalTok{ y\_),}
             \AttributeTok{color =} \StringTok{"gray30"}\NormalTok{, }\AttributeTok{alpha =} \FloatTok{0.5}\NormalTok{) }\SpecialCharTok{+}
  \FunctionTok{xlab}\NormalTok{(}\ConstantTok{NULL}\NormalTok{) }\SpecialCharTok{+}
  \FunctionTok{ylab}\NormalTok{(}\ConstantTok{NULL}\NormalTok{) }\SpecialCharTok{+}
  \FunctionTok{scale\_color\_brewer}\NormalTok{(}\AttributeTok{name =} \StringTok{"MCP Level"}\NormalTok{,}
                     \AttributeTok{palette =} \StringTok{"Set1"}\NormalTok{) }\SpecialCharTok{+}
  \FunctionTok{theme\_bw}\NormalTok{() }\SpecialCharTok{+}
  \FunctionTok{theme}\NormalTok{(}\AttributeTok{legend.position =} \StringTok{"bottom"}\NormalTok{, }
        \AttributeTok{legend.box.background =} \FunctionTok{element\_rect}\NormalTok{(}\AttributeTok{colour =} \StringTok{"black"}\NormalTok{, }\AttributeTok{linewidth =} \FloatTok{0.8}\NormalTok{))}
\end{Highlighting}
\end{Shaded}

\includegraphics{initial_analysis_files/figure-latex/KDE home range estimation-3.pdf}

\begin{Shaded}
\begin{Highlighting}[]
\FunctionTok{hr\_overlap}\NormalTok{(mcps,kdes)}
\end{Highlighting}
\end{Shaded}

\begin{verbatim}
## # A tibble: 4 x 2
##   levels overlap
##    <dbl>   <dbl>
## 1   1     0.419 
## 2   0.95  0.0782
## 3   0.75  0     
## 4   0.5   0
\end{verbatim}

\begin{enumerate}
\def\labelenumi{\alph{enumi}.}
\setcounter{enumi}{2}
\tightlist
\item
  Autocorrelated Kernel Density estimator of home range
\end{enumerate}

\begin{Shaded}
\begin{Highlighting}[]
\CommentTok{\# Without going into detail on the different CTMMs, we\textquotesingle{}ll demonstrate fitting}
\CommentTok{\# an aKDE with an Ornstein{-}Uhlenbeck model.}
\NormalTok{akdes }\OtherTok{\textless{}{-}} \FunctionTok{hr\_akde}\NormalTok{(trk\_108, }\AttributeTok{model =} \FunctionTok{fit\_ctmm}\NormalTok{(trk\_108, }\StringTok{"ou"}\NormalTok{), }\AttributeTok{levels =} \FloatTok{0.95}\NormalTok{)}
\end{Highlighting}
\end{Shaded}

\begin{verbatim}
## Default grid size of 35.3333333333333 seconds chosen for bandwidth(...,fast=TRUE).
\end{verbatim}

\begin{Shaded}
\begin{Highlighting}[]
\CommentTok{\# Get area in km\^{}2}
\FunctionTok{hr\_area}\NormalTok{(akdes, }\AttributeTok{units =} \ConstantTok{TRUE}\NormalTok{) }\SpecialCharTok{\%\textgreater{}\%} 
  \FunctionTok{mutate}\NormalTok{(}\AttributeTok{area =}\NormalTok{ units}\SpecialCharTok{::}\FunctionTok{set\_units}\NormalTok{(area, }\StringTok{"km\^{}2"}\NormalTok{))}
\end{Highlighting}
\end{Shaded}

\begin{verbatim}
## # A tibble: 3 x 3
##   level what         area
##   <dbl> <chr>      [km^2]
## 1  0.95 lci (0.95)  4861.
## 2  0.95 estimate    9858.
## 3  0.95 uci (0.95) 16588.
\end{verbatim}

\begin{Shaded}
\begin{Highlighting}[]
\CommentTok{\# plot the polygon isopleths by themeself}
\FunctionTok{plot}\NormalTok{(}\FunctionTok{hr\_isopleths}\NormalTok{(akdes)[}\DecValTok{1}\NormalTok{])}
\end{Highlighting}
\end{Shaded}

\includegraphics{initial_analysis_files/figure-latex/unnamed-chunk-2-1.pdf}

\begin{Shaded}
\begin{Highlighting}[]
\CommentTok{\# Plot the isopleth with data on top}
\FunctionTok{plot}\NormalTok{(akdes) }
\end{Highlighting}
\end{Shaded}

\includegraphics{initial_analysis_files/figure-latex/unnamed-chunk-2-2.pdf}

\begin{Shaded}
\begin{Highlighting}[]
\CommentTok{\# Custom plot with ggplot2 (thanks to Brian Smith for this code chunk)}

\FunctionTok{hr\_isopleths}\NormalTok{(akdes) }\SpecialCharTok{\%\textgreater{}\%} 
  \CommentTok{\# Make level a factor for discrete color scales}
  \CommentTok{\# Can control order and labels here}
  \FunctionTok{mutate}\NormalTok{(}\AttributeTok{level =} \FunctionTok{factor}\NormalTok{(level, }
                        \AttributeTok{levels =} \FunctionTok{c}\NormalTok{(}\StringTok{"1"}\NormalTok{, }\StringTok{"0.95"}\NormalTok{, }\StringTok{"0.75"}\NormalTok{, }\StringTok{"0.5"}\NormalTok{),}
                        \AttributeTok{labels =} \FunctionTok{c}\NormalTok{(}\StringTok{"100\%"}\NormalTok{, }\StringTok{"95\%"}\NormalTok{, }\StringTok{"75\%"}\NormalTok{, }\StringTok{"50\%"}\NormalTok{))) }\SpecialCharTok{\%\textgreater{}\%} 
  \FunctionTok{ggplot}\NormalTok{() }\SpecialCharTok{+}
  \FunctionTok{geom\_sf}\NormalTok{(}\FunctionTok{aes}\NormalTok{(}\AttributeTok{color =}\NormalTok{ level), }
          \AttributeTok{fill =} \ConstantTok{NA}\NormalTok{, }\AttributeTok{linewidth =} \DecValTok{2}\NormalTok{) }\SpecialCharTok{+}
  \FunctionTok{geom\_point}\NormalTok{(}\AttributeTok{data =}\NormalTok{ mcps}\SpecialCharTok{$}\NormalTok{data, }\FunctionTok{aes}\NormalTok{(}\AttributeTok{x =}\NormalTok{ x\_, }\AttributeTok{y =}\NormalTok{ y\_),}
             \AttributeTok{color =} \StringTok{"gray30"}\NormalTok{, }\AttributeTok{alpha =} \FloatTok{0.5}\NormalTok{) }\SpecialCharTok{+}
  \FunctionTok{xlab}\NormalTok{(}\ConstantTok{NULL}\NormalTok{) }\SpecialCharTok{+}
  \FunctionTok{ylab}\NormalTok{(}\ConstantTok{NULL}\NormalTok{) }\SpecialCharTok{+}
  \FunctionTok{scale\_color\_brewer}\NormalTok{(}\AttributeTok{name =} \StringTok{"MCP Level"}\NormalTok{,}
                     \AttributeTok{palette =} \StringTok{"Set1"}\NormalTok{) }\SpecialCharTok{+}
  \FunctionTok{theme\_bw}\NormalTok{() }\SpecialCharTok{+}
  \FunctionTok{theme}\NormalTok{(}\AttributeTok{legend.position =} \StringTok{"bottom"}\NormalTok{, }
        \AttributeTok{legend.box.background =} \FunctionTok{element\_rect}\NormalTok{(}\AttributeTok{colour =} \StringTok{"black"}\NormalTok{, }\AttributeTok{linewidth =} \FloatTok{0.8}\NormalTok{))}
\end{Highlighting}
\end{Shaded}

\includegraphics{initial_analysis_files/figure-latex/unnamed-chunk-2-3.pdf}

\begin{Shaded}
\begin{Highlighting}[]
\FunctionTok{hr\_overlap}\NormalTok{(kdes, akdes)}
\end{Highlighting}
\end{Shaded}

\begin{verbatim}
## # A tibble: 2 x 2
##   levels overlap
##    <dbl>   <dbl>
## 1   0.95   0.795
## 2   0.5    1.00
\end{verbatim}

\begin{enumerate}
\def\labelenumi{\arabic{enumi}.}
\setcounter{enumi}{2}
\item
  Explore the movement of all animals in the data set extracting a
  continuous covariate (for example elevation or distance to roads) to
  the data. Extract the covariate both at the end points and along the
  steps. Explore the differences between the two ways of extracting the
  data, for example fitting a linear regression with step length as
  response variable and the extracted variable as explanatory variable.
\item
  Fit a habitat selection function of your own choice (resource
  selection function at one scale or an (integrated) step-selection
  function) to the data using one covariate.
\end{enumerate}

\end{document}
